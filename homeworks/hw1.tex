\documentclass{article}

\usepackage{fullpage}
\usepackage{amsmath,amssymb,amsthm}
\usepackage{url}
\usepackage[colorlinks=true]{hyperref}

\newcommand\reals{\mathbb{R}}

\author{Ambuj Tewari}
\title{STATS 607A (Fall '14): Assignment 1\\
Due: Sep 24, 2014}
\date{Sep 15, 2014}


\begin{document}

\maketitle

\subsection*{Ways to earn extra credit}

\begin{itemize}
\item +1 for all real bugs in the already supplied code that you find and report to the instructor.
\item +1 for each problem where your code is in the top 10 percentile (top 5 students) in terms of running time.
\item +1 for each python script where Python style guide checker {\tt pep8} doesn't report any issues. You can test for any style guide issues yourself by typing
{\tt pep8 <python-script>} on the shell prompt.

\end{itemize}

\subsection*{How to turn in?}

Create a folder called {\tt stats607-fall2014} (case-sensitive) in your home directory on any of the Bayes machines. Make it publicly readable by typing:\\
{\tt chmod a+rx stats607-fall2014} \\
in your home directory. Then create a sub-directory called {\tt assignment\_one} within the 607 directory. Make that publicly readable too. We will try to find the
following 4 files in your home directory after the submission deadline:\\
{\tt assignment\_one\_kmeans.py} \\
{\tt assignment\_one\_optimization.py} \\
{\tt assignment\_one\_nbayes.py} \\
{\tt assignment\_one\_answers.pdf} \\

The first 3 files should be python script that run without any error message. The last file should be a PDF file with answers to all questions below. Make sure the 4 files
themselves are readable. You can check the permission of any of these 4 files by typing {\tt ls -l <filename>} and you should see {\tt -rw-r--r--} in the first field of the output.

\section{K-means (10 points)}

In this problem, we will implement the \href{http://en.wikipedia.org/wiki/K-means_clustering\#Standard_algorithm}{K-means algorithm}. Download a test dataset ``seeds" from:\\
\url{http://archive.ics.uci.edu/ml/datasets/seeds} \\
In the file {\tt seeds\_dataset.txt}, there are 210 instances of dimension 7 along with a categorical label (1, 2, or 3). Download an (incomplete!) python script by following the following
link:\\
\href{https://github.com/ambujtewari/stats607a-fall2014/blob/master/homeworks/assignment_one_kmeans.py}{\tt assignment\_one\_kmeans.py} \\
Click on the ``Raw" button on the top right and save the file as {\tt assignment\_one\_kmeans.py}.

Open the python script in your favorite text editors. You will see a bunch of places where a comment says {\tt TASK x.y(.z)}. These are the places where you have to supply your
own code.\\
{\bf Important:} Please {\em don't} modify the existing code and comments! We might use automated scripts to grade your code and not following this suggestion will break those scripts.

We will now briefly describe your tasks. If something is unclear, please don't hesitate to email the instructor and/or GSI.

\subsection{Reading in the data (1 point)}

We will read in the instances into two lists: {\tt instances} and {\tt labels}. The former will a list of 7-dimensional instance. An instance will itself be represented using a list of length 7.
The latter will be a list of the labels (note that the label occurs at the end of each line).

Task 1.1 has 2 subtasks: 1.1.1 and 1.1.2.

\subsection{Finding number of unique labels (1 point)}

Finish the definition of the function {\tt num\_unique\_labels} so that it finds out how many unique labels are there in its argument {\tt labels}. Make sure it returns 3 for the labels obtained
from the seeds dataset.

\subsection{Implementing K-means++ (2 points)}

\href{http://en.wikipedia.org/wiki/K-means\%2B\%2B}{K-means++} is an enhanced version of the classic K-means algorithm. It only differs from the classic algorithm in the way it initializes the $K$ centers.
Implement the function {\tt kmeans\_plus\_plus} so that this enhanced initialization is available by supplying the optional argument {\tt init} to {\tt cluster\_using\_kmeans}. Its default value is ``random" (in this case,
the initial centers are simply chosen at uniformly at random without replacement).

\subsection{Cluster assignment step of K-means (2 points)}

Implement the function {\tt assign\_cluster\_ids}.

Task 1.4 has 2 subtasks: 1.4.1 and 1.4.2.

\subsection{Center re-computation step of K-means (2 points)}

Implement the function {\tt recompute\_centers}.

Task 1.5 has 2 subtasks: 1.5.1 and 1.5.2.

\subsection{Running the script and discussing the output}

Once you have supplied all missing pieces of the code, run the script using:\\
{\tt python assignment\_one\_kmeans.py}

Answer the following questions:
\begin{description}
\item[Q. 1.1]
How does the kmeans clustering compare to the provided labels? On how many instances do they disagree?
\item[Q. 1.2]
How does the kmeans++ clustering compare to the provided labels? How does it compare to the kmeans clustering? Is there any difference at all?
\end{description}


\section{Simple Optimization}


\section{Naive Bayes}

\end{document}
